% 中文关键字
\newcommand{\keywordsCn}[1][\LaTeX 模版;合肥工业大学;教程]{#1}
% 中文摘要
\begin{abstract}
    本项目致力于开发一个高效的 \LaTeX 模板,旨在为合肥工业大学的同学们撰写毕业设计论文提供便利,从而免去繁琐的格式调整工作。相比于 Word,\LaTeX 在处理数学公式、实现精确严格的格式控制,以及管理参考文献方面展现出显著优势。重要的是,\LaTeX 的源文件为纯文本格式,这不仅方便使用版本控制工具进行管理,还极大地简化了协作写作过程。采用 \LaTeX 编写文档意味着可以将注意力集中在内容创作上,而无需担心格式问题。
    
    此外,本项目不仅提供一个模板,还包括一个全面的教程。我们将在模板中加入详尽的注释,以辅助同学们更好地理解和掌握 \LaTeX 的使用技巧。我们希望通过这个综合性的模板,同学们不仅能够学习到 \LaTeX 的操作方法,还能掌握一系列有价值的论文写作技巧。我们的目标是让这个模板成为一个完善的学习资源,帮助同学们高效地完成他们的毕业设计论文,同时提升写作和排版能力。
\end{abstract}
\addcontentsline{toc}{section}{摘\hspace{2em}要}

% 英文关键字
\newcommand{\keywordsEn}[1][\LaTeX\ Template, HFUT, Tutorial]{#1}
% 英文摘要
\begin{abstractEn}
    This project is committed to developing an efficient \LaTeX\ template, specifically designed to facilitate the students of Hefei University of Technology in writing their graduation design theses, thereby eliminating the need for tedious format adjustments. Compared to Word, \LaTeX\ offers significant advantages in handling mathematical formulas, achieving precise and strict format control, and in the management of references. Importantly, the source files of \LaTeX\ are in plain text format, which not only makes it convenient to manage using version control tools but also greatly simplifies the collaborative writing process. Writing documents with \LaTeX\ means that one can concentrate on content creation without the concern for formatting issues.
    
    In addition, this project offers not just a template but also an extensive tutorial. We plan to incorporate detailed annotations within the template to assist students in better understanding and mastering the use of \LaTeX. Through this comprehensive template, we aim for the students to not only learn the operational methods of \LaTeX\ but also to acquire a range of valuable thesis writing skills. Our goal is to make this template a complete learning resource, helping students efficiently complete their graduation design theses while enhancing their writing and typesetting skills.
\end{abstractEn}
\addcontentsline{toc}{section}{Abstract}